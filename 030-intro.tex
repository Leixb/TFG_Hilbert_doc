%! TEX root = **/010-main.tex
% vim: spell spelllang=en:

\section{Context and scope}%
\label{sec:context}

\subsection{Introduction and contextualization}

In 1900 David Hilbert posed a list of 23 problems in the field of mathematics
which were unsolved at the time.  This problem consists of two separate
problems, the first one regarding the relative positions of the branches of real
algebraic curves and the second one about the upper limit of limit cycles on two
dimensional vector fields and their relative positions. In this project we are
going to study the second part of 16th problem.

In particular, we study the number of limit cycles for vector fields of
polynomials of second degree:

\begin{align}
    \frac{dx}{dt} &= a_1x^2 + b_1xy + c_1y^2 + \alpha_1x + \beta_1y \\
    \frac{dy}{dt} &= a_2x^2 + b_2xy + c_2y^2 + \alpha_2x + \beta_2y
\end{align}

There have been various studies on the number of limit cycles for these kinds of
vector fields but so far the maximum number of cycles found is 4 \cite{kuznetsov_visualization_2013}. The aim of
this project is to search the parameter space for systems that have 4 or more
cycles to gain more insight on the nature of these equations.

To do so, we'll implement a parallel algorithm that solves ODE systems and
detects limit cycles for a wide range of parameters and points in the plane.
This code will be implemented in Julia programming language and will use CUDA
to be able to run it in the GPUs at the Minotaur cluster from the BSC.

\newcommand{\myparagraph}[1]{\paragraph{#1}\mbox{}\\}

\myparagraph{Hi there}

hello

\subsubsection{Context}
\subsubsection{Concepts}
\subsubsection{Problem to be solved}
\subsubsection{Stakeholders}

\subsection{Justification}
\subsubsection{Previous studies}

\subsection{Scope}
\subsubsection{Objectives and sub-objectives}
\subsubsection{Requirements}
\subsubsection{Potential obstacles and risks}

\subsection{Methodology and rigor}

\subsubsection{Methodology}

\subsubsection{Monitoring tools and validation}
