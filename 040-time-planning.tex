%! TEX root = **/010-main.tex
% vim: spell spelllang=en:

\section{Time planning}

The work-to-begin date is on February 9th and the delivery date on June which
gives around 18 weeks of time. I plan to work on the project approximately 35
hours a week, this gives around 630 hours in total working on the project.

\subsection{Description of the tasks}

\subsubsection{Task definition}

The definitions of the tasks that will be done throughout the project are
divided into 4 categories: planning, research, implementation and experimentation.

\paragraph{Project planning}

\begin{itemize}
    \item \textbf{Contextualization and project scope} Describe the project
        scope and its context. Giving an overview of the objective, other past
        studies on the topic and how this project is relevant.
    \item \textbf{Time planning} Organize the work to be done in granular tasks
        to estimate the time needed for each of them. With the time estimation
        create a realistic planning for the tasks completions so that the
        project can be finished in the appropriate time frame.
    \item \textbf{Budget and sustainability} Analyze the economic and
        environmental sustainability of the project.
    \item \textbf{Meetings} Each week a meeting with the tutor will be scheduled
        to ensure that the thesis is proceeding correctly and within the
        expected deadlines.
    \item \textbf{Integration into the final document} All these project
        planning tasks must be integrated into the final thesis memoir.
\end{itemize}

\paragraph{Research}

Given the nature of this thesis it is fundamental to have a solid understanding
of modern numerical computation techniques applied to solving Ordinary
Differential Equations and finding limit cycles. It is also primordial to know
how these techniques can be applied to CUDA and how to benchmark the programs to
detect bottlenecks and find whether or not the processing power of the GPU is
being used in its full potential.

\begin{itemize}
    \item \textbf{Research ODE solvers} There are lots of algorithm to solve
        ODEs which will have to be tried to find the ones that give the best
        results in our case.
    \item \textbf{Research how to find limit cycles} Find the best approach to
        detect limit cycles
    \item \textbf{Research CUDA} This task includes reading the official
        documentation as well as other sources on numerical methods applied to
        CUDA.
\end{itemize}

\paragraph{Practical Implementation}

The different algorithms researched must be implemented in order to experiment
with them and find the best ones to solve the problem.

\begin{itemize}
    \item \textbf{Program different methods to find limit cycles} As stated
        before various different methods will have to be implemented. This
        implementation can be initially without using CUDA (fully sequential).
    \item \textbf{Adapt de code to be run with CUDA} Once the methods to find
        limit cycles are properly implemented they must be adapted in order to
        be run with CUDA.
    \item \textbf{Test the program} In order to ensure that the code implemented
        is correct and the errors introduced in the computation fall within a
        reasonable distance of the real theoretical value some tests must be
        implemented. This includes also the implementation of small programs to
        visualize the results and interpret them.
\end{itemize}

\paragraph{Experimentation, analysis and conclusions}

Once there is a working prototype of the code the experimentation can begin.
There will be two major parts of the experimentation, the first one when the
various methods are tested with a relatively small number of cases to find the
best one (taking into account both the speed and accuracy of the calculations)
and the final experiment when the best method is run with a much bigger search
space and from which the results can be analysed.

\begin{itemize}
    \item \textbf{Comparison experiment}
        \begin{itemize}
            \item \textbf{Select search space} Decide how many different
                parameters will be used on the experiment. It should be a big
                enough search space so as to have variety on the systems but not
                big enough that it takes too much time to benchmark and slows
                down the development.
            \item \textbf{Select benchmarks} The benchmarks must be run in
                similar conditions and we must decide what metrics will be
                considered (runtime, GFLOPS, accuracy, \dots).
            \item \textbf{Further optimize the methods} When running the
                benchmarks we may find some possible improvements on the
                original implementations which can then be improved and tested
                again.
            \item \textbf{Analyse results and decide best method} With the
                results of the experiments we can decide on which method is the
                best and should therefore be used in the final experiment.
        \end{itemize}
    \item \textbf{Final experiment}
        \begin{itemize}
            \item \textbf{Select search space} Given the results of the previous
                experiments, estimate the runtime as a function of the search
                space and select a search space as big as possible given the
                available computing time.
            \item \textbf{Analyse results} Once the final experiment finishes
                the results can be analysed in search of systems with
                interesting number of limit cycles.
        \end{itemize}
    \item \textbf{Conclusions} Analyse the results of both experiments and
        provide a conclusion.
\end{itemize}

\pagebreak
\subsubsection{Summary of the tasks}

\begin{table}[H]
    \centering
    \caption{Summary of tasks}
    \begin{tabular}{llrc}
        \toprule
        \thead{ID} & \thead[l]{Task} & \thead{Time (h)} & \thead{Depend.} \\
        \midrule
    \texttt{P} & \textbf{Project planning} & \textbf{60} & \\
        \texttt{P0} & - Contextualization and project scope & 10 & \\
        \texttt{P1} & - Time planning & 10 & \\
        \texttt{P2} & - Budget and sustainability & 10 & \\
        \texttt{P3} & - Meetings & 18 & \\
        \texttt{P4} & - Integration into final document & 12 & \\

        \addlinespace[0.5em]
    \texttt{R} & \textbf{Research} & \textbf{150} & \\
        \texttt{R0} & - ODE solvers & 50 & \\
        \texttt{R1} & - Limit cycles & 50 & \\
        \texttt{R2} & - CUDA & 50 & \\

        \addlinespace[0.5em]
    \texttt{I} & \textbf{Implementation} & \textbf{200} & \\
        \texttt{I0} & - Different methods to find limit cycles & 100 & \texttt{R0,R1} \\
        \texttt{I1} & - Adapt to CUDA & 50 & \texttt{I0,R2} \\
        \texttt{I2} & - Tests & 50 & \\

        \addlinespace[0.5em]
    \texttt{EC} & \textbf{Experimentation (comparison)} & \textbf{130} & \texttt{I} \\
        \texttt{EC0} & Select search space & 5 & \\
        \texttt{EC1} & Select benchmarks & 5 & \\
        \texttt{EC2} & Further optimize the methods & 100 & \\
        \texttt{EC3} & Analyse the results and decide the best method & 20 & \\

    \addlinespace[0.5em]
    \texttt{EF} & \textbf{Experimentation (final)} & \textbf{25} & \texttt{EC} \\
        \texttt{EF0} & Select search space & 5 & \\
        \texttt{EF1} & Analyse the results & 20 & \\

    \addlinespace[0.5em]
        \texttt{C} & \textbf{Conclusions} & \textbf{10} & \texttt{EC,EF} \\
        \texttt{D} & \textbf{Documentation} & \textbf{60} & \\
        \texttt{O} & \textbf{Oral exposition} & \textbf{10} & \\
        \bottomrule
    \end{tabular}
\end{table}
% table

\pagebreak
\subsubsection{Resources}

\paragraph{Human resources}

The main human resource of the thesis is the researcher.  There is also the
director Grigori Astrakharchik which mentors the researcher and the GEP Tutor
Eguiguren Huerta Marcos in charge of correcting the project management part.

\paragraph{Material resources}

The main resources needed for this project are previous papers and books on the
topics of ODEs, limit cycles and CUDA. For the implementation an execution of
the code there following resources will be used:

\begin{itemize}
    \item \textbf{VCS}: \texttt{git} will be used as a version control
        system~(VCS) and the code will be hosted on \emph{GitHub} for easy
        collaboration with the director.
    \item \textbf{\LaTeX}: To format the document \LaTeX will be used. The
        document will be hosted on \emph{Overleaf} to allow easier collaboration
        and due to the fact that it provides \emph{github} integration.
    \item \textbf{Atenea}: To communicate with the GEP tutor.
    \item \textbf{Computers}: My own personal computer with an \emph{RTX2060} will be
        used to develop the code and test it. The final versions will will run
        on a server from the department of physics at UPC with a \emph{Nvidia
        Titan V} and \emph{Nvidia Titan Xp} GPUs.
    \item \textbf{Julia}: The Julia programming language will be used to write
        the code and visualize the results.
\end{itemize}

%! TEX root = **/010-main.tex
% vim: spell spelllang=en:

\newgeometry{top=0.5cm, bottom=0.5cm, left=0.5cm, right=0.5cm}
\thispagestyle{empty}
\begin{landscape}
\subsection{Gantt chart}

\begin{figure}[H]
\centering
\resizebox{28cm}{!}{%
\begin{ganttchart}[
vgrid={*{6}{draw=none}, dotted},
x unit=.40cm,
y unit title=1cm,
y unit chart=1cm,
    time slot format=isodate
    ]{2021-02-09}{2021-06-07}
\gantttitlecalendar{month=name, day}
\ganttnewline

\ganttgroup{Project planning}{2021-02-09}{2021-03-22}
\ganttnewline
\ganttbar{Contextualization and scope}{2021-02-23}{2021-03-02}
\ganttnewline
\ganttbar{Time planning}{2021-03-02}{2021-03-09}
\ganttnewline
\ganttbar{Budget and sustainability}{2021-03-09}{2021-03-16}
\ganttnewline
\ganttbar{Integration into final document}{2021-03-16}{2021-03-22}
\ganttnewline

\ganttgroup{Research}{2021-02-09}{2021-04-01}
\ganttnewline
\ganttbar{ODE Solvers}{2021-02-09}{2021-03-02}
\ganttnewline
\ganttbar{Limit cycles}{2021-02-16}{2021-03-09}
\ganttnewline
\ganttbar{CUDA}{2021-03-09}{2021-03-22}
\ganttnewline

    % \texttt{R} & \textbf{Research} & \textbf{150} & \\
    %     \texttt{R0} & - ODE solvers & 50 & \\
    %     \texttt{R1} & - Limit cycles & 50 & \\
    %     \texttt{R2} & - CUDA & 50 & \\

\ganttgroup{Implementation}{2021-02-22}{2021-05-02}
\ganttnewline
\ganttbar{Find limit cycles}{2021-02-22}{2021-04-01}
\ganttnewline
\ganttbar{Adapt to CUDA}{2021-04-01}{2021-05-02}
\ganttnewline
\ganttbar{Tests}{2021-03-09}{2021-05-02}
\ganttnewline

    %     \addlinespace[0.5em]
    % \texttt{I} & \textbf{Implementation} & \textbf{200} & \\
    %     \texttt{I0} & - Different methods to find limit cycles & 100 & \texttt{R0,R1} \\
    %     \texttt{I1} & - Adapt to CUDA & 50 & \texttt{I0,R2} \\
    %     \texttt{I2} & - Tests & 50 & \\

\ganttgroup{Experimentation (comparison)}{2021-04-05}{2021-05-16}
\ganttnewline
\ganttbar{Select search space}{2021-04-05}{2021-04-06}
\ganttnewline
\ganttbar{Select benchmarks}{2021-04-05}{2021-04-06}
\ganttnewline
\ganttbar{Further optimizations}{2021-04-07}{2021-05-01}
\ganttnewline
\ganttbar{Analyse results}{2021-05-01}{2021-05-16}
\ganttnewline

    %     \addlinespace[0.5em]
    % \texttt{EC} & \textbf{Experimentation (comparison)} & \textbf{130} & \texttt{I} \\
    %     \texttt{EC0} & Select search space & 5 & \\
    %     \texttt{EC1} & Select benchmarks & 5 & \\
    %     \texttt{EC2} & Further optimize the methods & 100 & \\
    %     \texttt{EC3} & Analyse the results and decide the best method & 20 & \\

\ganttgroup{Experimentation (final)}{2021-05-16}{2021-05-30}
\ganttnewline
\ganttbar{Select search space}{2021-05-16}{2021-05-17}
\ganttnewline
\ganttbar{Run experiment}{2021-05-17}{2021-05-25}
\ganttnewline
\ganttbar{Analyse results}{2021-05-25}{2021-05-30}
\ganttnewline

    % \addlinespace[0.5em]
    % \texttt{EF} & \textbf{Experimentation (final)} & \textbf{25} & \texttt{EC} \\
    %     \texttt{EF0} & Select search space & 5 & \\
    %     \texttt{EF1} & Analyse the results & 20 & \\

\ganttgroup{Project Documentation}{2021-02-09}{2021-06-07}
\ganttnewline
\ganttbar{Documentation}{2021-02-09}{2021-06-07}
\ganttnewline
\ganttbar{Conculsions}{2021-05-25}{2021-06-04}
\ganttnewline
\ganttbar{Oral exposition}{2021-06-01}{2021-06-07}
\ganttnewline
    % \addlinespace[0.5em]
    %     \texttt{C} & \textbf{Conclusions} & \textbf{10} & \texttt{EC,EF} \\
    %     \texttt{D} & \textbf{Documentation} & \textbf{60} & \\
    %     \texttt{O} & \textbf{Oral exposition} & \textbf{10} & \\

\end{ganttchart}
}

\caption{Gantt chart}%
\label{fig:gantt}
\end{figure}

\end{landscape}
\restoregeometry


\subsection{Risk Management}

% todo link with intro section

    \subsubsection{Project deadline}
    There is the possibility that the initial estimation made of the time it takes to
    complete each task was wrong due to unexpected difficulties. Therefore it's
    crucial that there is enough time to accommodate possible delays and that
    these setbacks are detected immediately and taken into account (Possibly
    redoing the initial time planning).

    \subsubsection{Computational power}
    The program will be run in a server of the department of physics of UPC.
    Although it should not be a problem in the event of not being able to use
    that computing power the code can be run locally on my own GPU (this will
    take much longer time).

    \subsubsection{Inexperience on the field}
    Sine I have limited experience with both CUDA and numerical computation of
    ODEs there is a significant amount of hours dedicated to studying the
    concepts and researching numerical methods.
