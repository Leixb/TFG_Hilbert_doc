%! TEX root = **/010-main.tex
% vim: spell spelllang=en:

\section{Budget}%
\label{sec:budget}

\subsection{Staff costs}%
\label{sub:staff}

Although the tasks in the project are going to be performed by me and the tutors
I will break them down into different roles in order to estimate the cost of the
project. We will have a \textbf{Project manager} responsible of planning the
project. A \textbf{Junior researcher} that will study the different
computational methods and techniques needed to perform the calculations in the
project. The \textbf{Junior developer} will implement the code according to the
instructions given by the researcher and this code will be tested by a
\textbf{Tester} to ensure that the implementation is correct. Finally a
\textbf{Technical writer} will compile all the results obtained and write the
final document.

Using data from \url{https://www.payscale.com} we can estimate the cost of the
different roles that take part in the project as shown in~\cref{tab:pay}. With
these estimations we can calculate the overall cost of the staff in the project
which is shown in~\cref{tab:cost}.

\begin{table}[H]
    \centering
    \caption{Cost per hour of different roles}\label{tab:pay}
    \begin{tabular}{cc}
        \toprule
        Role & Cost (€/h) \\
        \midrule
        Project manager & 23 \\
        Junior developer & 15 \\
        Junior researcher & 22 \\
        Tester & 8 \\
        Technical writer & 13 \\
        \bottomrule
    \end{tabular}
\end{table}

\begin{table}[H]
    \centering
    \caption{Total cost of tasks}\label{tab:cost}
    \begin{tabular}{lrcr}
        \toprule
        \textbf{Task} & \textbf{Time (h)} & \textbf{Role} & \textbf{Cost (€)} \\
        \midrule
    \textbf{Project planning} & \textbf{60} & PM & \textbf{1380} \\
        % - Contextualization and project scope & 10 & \\
        % - Time planning & 10 & \\
        % - Budget and sustainability & 10 & \\
        % - Meetings & 18 & \\
        % - Integration into final document & 12 & \\

        \addlinespace[0.5em]
        \textbf{Research} & \textbf{150} & JR & \textbf{3300}\\
        % - ODE solvers & 50 & \\
        % - Limit cycles & 50 & \\
        % - CUDA & 50 & \\

        \addlinespace[0.5em]
        \textbf{Implementation} & \textbf{200} & JD,T & \textbf{2650} \\
        - Different methods to find limit cycles & 100 & JD & 1500 \\
        - Adapt to CUDA & 50 & JD & 750 \\
        - Tests & 50 & T & 400 \\

        \addlinespace[0.5em]
        \textbf{Experimentation (comparison)} & \textbf{130} & JR,JD & \textbf{2160} \\
        - Select search space & 5 & JR & 110 \\
        - Select benchmarks & 5 & JR & 110 \\
        - Further optimize the methods & 100 & JD & 1500 \\
        - Analyse the results and decide the best method & 20 & JR & 440 \\

    \addlinespace[0.5em]
        \textbf{Experimentation (final)} & \textbf{25} & JR & \textbf{550} \\
        % Select search space & 5 & \\
        % Analyse the results & 20 & \\

    \addlinespace[0.5em]
        \textbf{Conclusions} & \textbf{10} & W & \textbf{130} \\
        \textbf{Documentation} & \textbf{60} & W & \textbf{780}\\
        \textbf{Oral exposition} & \textbf{10} & W & \textbf{130} \\
    \addlinespace[1em]
        \textbf{Total} & & & \textbf{11080} \\

        % 1380
        % 3300
        % 2650
        % 2160
        % 550
        % 130
        % 780
        % 130
        \bottomrule
    \end{tabular}
\end{table}

% \begin{table}[H]
%     \centering
%     \caption{Total cost of the staff}
%     \begin{tabular}{cc}
%         \toprule
%         Role & Cost (€) \\
%         \midrule
%         Project manager & 23 \\
%         Junior developer & 15 \\
%         Junior researcher & 22 \\
%         Tester & 8 \\
%         Technical writer & 13 \\
%         Total & \\
%         \bottomrule
%     \end{tabular}
% \end{table}


\pagebreak
\subsection{Generic costs}

% Other than the staff costs, there are other cost that must be taken into
% account.

\subsubsection{Amortization of the resources}

I will work approximately 3.7 hours per day during 130 days. Most of the time on
my Lenovo laptop (80\%) and the rest on an HP laptop which is more lightweight
and can be carried around easily. Using the formula to compute the
amortization~(\cref{eq:amort}) we obtain the \cref{table:amort} which shows the
amortization of the hardware used.

\begin{equation}\label{eq:amort}
    \text{Amortization} = \text{Price} \times \frac{1}{\text{Years of use}}
    \times \frac{1}{\text{Days of work}} \times \frac{1}{\text{Hours per day}}
    \times \text{hours used}
\end{equation}

\begin{table}[H]
    \centering
    \caption{Amortization of hardware}\label{table:amort}
    \begin{tabular}{crrr}
        \toprule
        Hardware & Cost (€) & Life expectancy (years) & Amortization (€) \\
        \midrule
        Lenovo laptop & 1.400 & 6 & 211.89 \\
        HP laptop & 600 & 4 & 28.38\\
        \midrule
        Total & 2.000 &   & 240.27 \\
        \bottomrule
    \end{tabular}
\end{table}

\subsubsection{Indirect costs}

A part from the hardware costs of the laptops there are more indirect costs that
must be considered:

\begin{itemize}
    \item \textbf{Internet:} With an internet cost of 100€ per month during 5
        months working 3.7 hours per day the total is: $100€ \cdot 5 \cdot
        \dfrac{3.7}{24} = 77.08€$.
    \item \textbf{Electricity} Given a cost of $0.1270€/kWh$ and the fact that
        my Lenovo laptop consumes $230W$ at peak performance (which should
        happen rarely) we can estimate the electricity cost as:
        $0.1270€/kWh \cdot 0.230kW \cdot 0.5 \cdot 630h = 9.21€$.
\end{itemize}

In total there are $86.29€$ of indirect costs that added to the hardware
amortization gives $326.56€$ of generic costs.

\subsection{Contingency}

During the development of the project unforeseen events may occur that may
impact our budget. Therefore a 15\% increase on the total cost will be added as
a contingency margin. Given that the CPA is $11,080€$ and the generic costs are
$326.56€$ for a total of $11,406.56€$. The contingency budget is then $1710.99€$.

\subsection{Incidental costs}

The following table estimates the cost of the different incidents that may
impact the project given their estimated cost and the risk that they happen.

\begin{table}[H]
    \centering
    \caption{Incidental costs}\label{tab:inc}
    \begin{tabular}{lrrr}
        \toprule
        Incident & Estimated Cost (€) & Risk (\%) & Cost (€) \\
        \midrule
        Project deadline & 500 & 25 & 125 \\
        Computational power & 100 & 50 & 50 \\
        Inexperience on the field & 300 & 25 & 75 \\
        \addlinespace[0.5em]
    \textbf{Total} & \textbf{900} & & \textbf{250} \\
        \bottomrule
    \end{tabular}
\end{table}


\subsection{Final budget}

\begin{table}[H]
    \centering
    \caption{Final budget}\label{tab:pay}
    \begin{tabular}{lr}
        \toprule
        Activity & Cost (€) \\
        \midrule
        CPA & 11,080.00 \\
        GC & 326.56 \\
        Contingency & 1,710.98 \\
        Incidental cost & 250.00 \\
        \addlinespace[0.5em]
        \textbf{Total} & \textbf{13,367.54} \\
        \bottomrule
    \end{tabular}
\end{table}

\pagebreak
\subsection{Management control}

In the previous section there is an estimation of the budget and its potential risks and
incidents along with their impact. We also need a model to detect the
deviation on these initial budget estimations. Upon finishing a task $t$, we will
calculate its deviation:

\begin{equation}
    d_t = E_t - R_t
\end{equation}

Where:

\begin{itemize}
    \item $E_t = $ \textbf{Estimated cost} of the task on the initial budget
        plan
    \item $R_t = $ \textbf{Real cost} of the task when finished. Here we have to
        recalculate the $CPA$, $GC$, contingency and incidents for the task.
    \item $d_t = $ \textbf{Deviation from initial cost:} this estimates how much
        we have deviated from the original budget for each task.
\end{itemize}

If $d_t$ is negative I will reallocate the extra money for future incidents, if
it's positive some of the funds for contingency will be used to cover the costs.
If the contingency funds are not enough the whole budget planning will be
readdressed.

\pagebreak
\section{Sustainability}%
\label{sec:sustainability}

\subsection{Self-assessment}

Before starting the sustainability assessment I did not know how many indicators
and factors must be taken into account when doing a sustainable project. I had
not considered the impact that project a part from the immediate actors involved
in it. Through this assessment I realized how the project can impact the
environment, the society and the economy in ways I had not thought about before.

\subsection{Environmental impact}

\noindent\textbf{Regarding PPP, Have you estimated the environmental impact of
undertaking the project? Have you considered how to minimise the impact, for
example by reusing resources?}

The main impact of this project on the environment is the use of computer
resources and therefore electric power to do the complex calculations needed.
The aim of the project is not only to develop a program that works but that it
also uses the least amount of resources so that it does not waste computing
power. In order to avoid wasting resources, only the final version will be run
on a big search space and during development the tests will be on a much reduced
search space.

\noindent\textbf{Regarding the life expectancy, How is the problem that you wish
to address resolved currently (state of the art)? In what ways will your
solution environmentally improve existing solutions?}

Currently there is not much work on the research of the number of limit cycles
in second degree ODEs and the current approaches rely on heady computations in
order to calculate the cycles. With this new approach the aim is to search with
a faster implementation although less accurate which will hopefully find
interesting systems that can then be analyzed in more detail.


\pagebreak
\subsection{Economy}

\noindent\textbf{Regarding PPP, Have you estimated the cost of undertaking the
project (human and material resources)?}

In~\cref{sec:budget} there is a description of the cost of the project taking
into account human and material resources as well as potential risks and
contingencies.

\noindent\textbf{Regarding the life expectancy, How is the problem that you wish
to address resolved currently (state of the art)? In what ways will your
solution economically improve existing solutions?}

As stated on the Sustainability section, if a faster more efficient method to
identify limit cycles is implemented it will reduce the amount of computational
power needed and therefore the resources.


\subsection{Social}

\noindent\textbf{Regarding PPP, What do you think undertaking the project has
contributed to you personally?}

The project enables me to research on various topics that interest me and has
given me a new view on the applications of GPUs for scientific research.

\noindent\textbf{Regarding the life expectancy, How is the problem that you wish
to address resolved currently (state of the art)? In what ways will your
solution socially improve (quality of life) existing Is there a real need for
the project?}

The nature of this project as a research project on the field of dynamical
systems does not have a direct impact on the society as a whole. However if
interesting results are obtained it could bring some insight into Hilbert 16th
problem which has remained unsolved for more than a century.
