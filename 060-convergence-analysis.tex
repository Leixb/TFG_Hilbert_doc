%! TEX root = **/010-main.tex
% vim: spell spelllang=en:

\section{Initial implementation}%
\label{sec:initial_implementation}

Before implementing the program in CUDA we first need to implement a sequential
version of the program. This sequential code allows easier debugging of the
implementation and provides a basis to compare the speedup obtained with CUDA.

As discussed previously, several different Runge Kutta based methods of different order
are going to be implemented and tested.

% Julia -> C

% Sample code

\section{Convergence analysis}%
\label{sec:convergence}

Once the initial sequential code is implemented we must analyze the results
obtained to verify the correctness of the program. Therefore a convergence
analysis was performed in which we compared the results obtained with our
implementation to the theoretical results of a well known ODE.

We used the system shown in \cref{eq:circle} for which all trajectories are
circles with center $(0,0)$ and period $2\pi$. For each method we computed the
error after one loop with different time steps ($\Delta t$)

In \cref{fig:error_cycle} we can observe how the error of all the methods
in relation to $\Delta t$ correlate to the numerical order of the method.
Additionally, for the higher order methods there is a point in which the
numerical error of the method is irrelevant since its less than the precision of
\texttt{double}, therefore there is no need for $\Delta t < \dfrac{2\pi}{10^4}$
for the Runge-Kutta of order 4.

In \cref{fig:error_pi} we computed the error using values of $\Delta t$ non commensurate
with the period of the circle and interpolated the period using a second degree
polynomial. The results are quite different to \cref{fig:error_cycle} since
Euler and Midpoint methods appear to have order 2 and RK3, RK4, RKF45 order 3.
This is probably due to the polynomial interpolation used and other
interpolation methods should be considered to obtain better results.

\begin{align}\label{eq:circle}
    \frac{dx}{dt} &= -y \nonumber \\
    \frac{dy}{dt} &= x
\end{align}

\begin{figure}[H]
    \centering
    \includegraphics[width=1.0\textwidth]{figures/plots/error_analysis/error_cycle.tikz}
    \caption{Integration error after one period with commensurate $\Delta t$}%
    \label{fig:error_cycle}
\end{figure}

\begin{figure}[H]
    \centering
    \includegraphics[width=1.0\textwidth]{figures/plots/error_analysis/error_pi.tikz}
    \caption{Error on the period estimation using interpolation}%
    \label{fig:error_pi}
\end{figure}
