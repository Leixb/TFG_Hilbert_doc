%! TEX root = **/010-main.tex
% vim: spell spelllang=en:

\section{Convergence analysis}%
\label{sec:convergence}

Once the sequential code is implemented, we must analyze the results obtained to
verify the correctness of the program. Therefore, a convergence analysis was
performed in which we compared the results obtained with our implementation to
the theoretical results of a well known ODE.

\begin{align}\label{eq:circle}
    \frac{dx}{dt} &= -y \nonumber \\
    \frac{dy}{dt} &= x
\end{align}
%{\bf what is this???}

We used the test system \cref{eq:circle} for which all trajectories are known to have a circular shape with center $(0,0)$ and a period of $T=2\pi$. For each method we found the numerical trajectory
${\bf r}(t)$ which depends on the specific method used and the timestep.
%computed the error after one loop with different time steps ($\Delta t$).
The error is calculated as the absolute value of the displacement after one period $T$,
\begin{equation}
\varepsilon = |{\bf r}(T)-{\bf r}(0)|
\label{Eq:epsilon:T}
\end{equation}

In \cref{fig:error_cycle} we show how the error depends on the timestep $\Delta t$ and on the specific integration scheme.
This allows to verify the consistency in the order of the integration scheme. For order $p$ of accuracy, the error scales as
\begin{equation}
\varepsilon = {\cal O}((\Delta t)^p)
\end{equation}
As expected, the higher-order methods have higher order of accuracy.
%of all the methods in relation to  correlate to the numerical order of the method.
Additionally, for the higher order methods there is a point in which the numerical error of the method becomes irrelevant since it is less than the {\em machine epsilon}.
The specific value of the machine epsilon depends on the type of variable, CPU architecture and compiler.
In our case using the precision of \texttt{double} we find that for the Runge-Kutta of order 4 the smallest useful timestep is $\Delta t \approx \dfrac{2\pi}{10^4}$.
A further decrease in the timestep actually worsens the accuracy as this implies a summation of a larger number of values, thus increasing the importance of the rounding error.

The error~(\cref{Eq:epsilon:T}) was calculated by taking into account the exact value of the period $T$. In a more general situation, the period might not be known. Thus we perform a different study in which the error is estimated as the difference in the maximal $x^{\max}$ position between two consequent periods, $x^{\max}_i$ and $x^{\max}_{i+1}$
\begin{equation}
\varepsilon = |x^{\max}_{i+1} - x^{\max}_{i}|
\label{Eq:epsilon:max}
\end{equation}
%{\bf check if you  agree ???}
We computed the error using values of $\Delta t$ non commensurate with the period of the circle and interpolated the period using a second degree polynomial. The results are shown in \cref{fig:error_pi} and are quite different to \cref{fig:error_cycle} since Euler and Midpoint methods appear to have order 2 and RK3, RK4, RKF45 order 3. This is probably due to the polynomial interpolation used and other interpolation methods should be considered to obtain better results.


\begin{figure}[H]
    \centering
    \includegraphics[width=1.0\textwidth]{figures/plots/error_analysis/error_cycle.tikz}
    \caption{Integration error after one period with commensurate $\Delta t$, estimated according to \cref{Eq:epsilon:T}.}
    \label{fig:error_cycle}
\end{figure}

\begin{figure}[H]
    \centering
    \includegraphics[width=1.0\textwidth]{figures/plots/error_analysis/error_pi.tikz}
    \caption{Error on the period estimation using interpolation, estimated according to \cref{Eq:epsilon:max}}
    \label{fig:error_pi}
\end{figure}
