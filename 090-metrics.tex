%! TEX root = **/010-main.tex
% vim: spell spelllang=en:

\section{Metrics}%
\label{sec:metrics}

As discussed in previous sections, there is no need to save the full trajectory in order to detect limit cycle candidates; we can evaluate the trajectory's characteristics in-place.

To do so we have to detect special points in the trajectory that we can use to detect if we are in a cycle. These points must be easy to identify (computationally cheap) and should be present in all trajectories. By comparing the values of these points at each loop of the cycle we can estimate the behaviour of the trajectory.

\paragraph{Zero crossings:} can be computed by detecting the change of sign
during the computation. Although they are really cheap and are commonly used on complex ODE systems, there is no guarantee that cycles will cross axis.

\paragraph{Inflection points:} these correspond to the points where the ratio of change of the function changes sign (maxima, minima or stationary points). Given that the integration methods used compute the result as the previous value plus a change, we can easily detect changes in sign with almost no overhead. Moreover, all cycles will have at least a four local extrema.

\paragraph{Interpolation:} although we can detect if we passed an inflection point with almost no overhead, we must perform some kind of interpolation to obtain an accurate value within our tolerances. To do so, we can perform interpolation using 3 neighbouring points or perform additional integration steps to find the inflection point numerically. Applying polynomial interpolation requires saving at least 3 point for each trajectory, interpolating them and computing the vertex. The second option (root finding) is potentially more costly since it involves additional integration steps but should produce better results.

\pagebreak

\subsection{Rate of change of extrema}

Comparing the value of the initial 4 local extrema with the local extrema of the second loop of the cycle we can obtain a rate of change of the trajectory. For instance, given the values of the 4 local extrema on obtained on the first loop: $x_{\min}, x_{\max}, y_{\min}, y_{\max}$ and their counterparts obtained on the second loop: $x'_{\min}, x'_{\max}, y'_{\min}, y'_{\max}$ we can compute the ratio between each of them to obtain their rate of change. If the ratio is less than 1, the trajectory is \emph{decreasing}, if it is greater than 1 it is \emph{increasing} and if it is 1 (within an adequate tolerance) the trajectory is a cycle.

There is one small caveat with this approach which is that the ratios between the 4 different extrema are not comparable, that is, in some cases the maxima or minima are various orders of magnitude apart. This has to be taken into consideration when evaluating the values used for the tolerance when searching the points with ratio 1.

Traditionally with CPU computation one initial point is taken and evaluated through many steps until the trajectory reaches an stable cycle. Instead, the advantage of GPU computation is that one can take a massive amount of points, evaluate only the very few first steps (until 2 loops are completed) and quickly find limit cycle candidates.

\subsection{Other metrics}

A part from the rate of change in local extrema, an evaluation of the rate of change in the loop period was also attempted but it did not give satisfying results.

\subsection{Distinguishing different cycles}

The rate of change of extrema allows us to find areas in the function with trajectories with ratio 1, 
{\bf not clear which ratio is meant, a way out is to introduce an equation with the definition of the ratio ???}
which are candidates to be limit cycles. Now the problem is how we can classify this areas into different limit cycle trajectories. Two approaches were tested:

\paragraph{Computer vision techniques}

{\bf the title is not clear in its relation to the text??? change the title or explain better}
The final result of our computation is a matrix with ones and zeros indicating the coordinates close to trajectories with rate of change 1. A priori it should be easy to run this matrix through some form of \emph{BFS} able to find the different groups. However, this presents 2 difficulties: there may be areas where there is not enough \emph{resolution} and the line can be partially broken. Additionally some trajectories may pass really close to each other and be classified as the same group.

\paragraph{Clustering}
Given that we are computing the ratio of 4 local extrema points, we can use these same extrema values to analyse the characteristics of the trajectory. To do so, a part from returning the ratio of the extrema points, the 4 extrema points themselves will also be returned. With this information we have 4 points defining a cycle which we can classify using any clustering technique. Three different clustering methods where tested:

\paragraph{Hierarchical clustering}

Using hierarchical clustering we obtained correct clusters with the system of \cite{kuznetsov_visualization_2013}. However, the this algorithm does not scale well when lots of points are used. ($\mathcal{O}(n^2)$).

\paragraph{Kmeans clustering}

As with hierarchical clustering, with \emph{Kmeans} we can use the 4 values of local extrema when computing the clustering. 
{\bf provide an equation and refer to it later ???}
With \emph{Kmeans} we must specify the number of clusters the search, since the number of possible limit cycles is small (at most 5) we can compute the 5 possible number of clusters and evaluate which method gives better results. The runtime complexity of \emph{Kmeans} is $\mathcal{O}(n)$.

Some tests where done with hierarchical clustering but the results where similar to the ones obtained using \emph{Kmeans}

\paragraph{Jenks natural breaks optimization}

This algorithm is equivalent to \emph{Kmeans} but using only one variable.
{\bf provide an equation to introduce the quantity more formally}
We computed the method with different number of clusters (1 to 5) and determined for each of the 4 variables which number of clusters gave the bests results. Most of the times all 4 variables gave the same number of results, indicating that one could suffice.
