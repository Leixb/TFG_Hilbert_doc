%! TEX root = **/010-main.tex
% vim: spell spelllang=en:

\section{Compactification and Stiffness}%
\label{sec:compact-stiff}

One problem that arises when trying to find limit cycles is where to look for
them in the space. Since the $\mathcal{R}^2$ plane is infinite we must decide
what boundaries of the plane to evaluate. There are 2 methods to solve this
issue: compactification (compacting the infinite field into a finite one through
mathematical constructs) and analytical analysis of singular points interest.

\paragraph{Compactification}
There are various methods of compactification that can be used to study
limit cycles, one of these is the Poincaré, Bendixon or
Poincaré Lyupanov compactification
\cite{poincare_sur_1891,bendixson_sur_1901,dumortier_poincare_2006,noauthor_fig_nodate}
among others.  These methods reduce the infinite field into an semi-sphere that
that can be projected onto a plane for easier visualization. For simplicity, we
compactified the plane using the tangent function: a point $x$ in the infinite
space can be mapped to $\alpha \in \left(-\frac{\pi}{2}, \frac{\pi}{2}\right)$
such that: $x = \tan(\alpha)$. This produces great distortion on the plane
making all trajectories seem squares and values of $x$ greater than

\paragraph{Singular points}
Performing analytical analysis to find points of interest is quite complex and
requires vast knowledge on the field of ODEs and limit cycles. Moreover, to
perform analytical computation special software to handle symbolic maths is
required (like Maple \cite{noauthor_maple_nodate}). Ideally these points could
be computed beforehand using normal CPU computation and use them to decide the
areas that need analysis using our program. However it is known that all limit
cycles in a quadratic system contain a node \cite{cherkas_quadratic_2003}. And
in the system we are analysing $(0,0)$ will always be a node, therefore limit
cycles should appear around the origin. There may be more nodes
depending on the parameters of the system, but we know that the area around the
origin is a good candidate.

\paragraph{Stiffness}
Another problem when numerically integrating ODE systems are the so called
\emph{stiff} systems where traditional \emph{Runge-Kutta} integration methods
struggle due to the numerical instability of the system.

%TODO
