%! TEX root = **/010-main.tex
% vim: spell spelllang=en:

\section{Julia ABI}%
\label{sec:abi}

\paragraph{CUDA limitations}

One of the main problems with the current implementation is that changing the system
involves creating a new \emph{CUDA} function and recompiling the library, this makes
using the program as a library quite challenging since the use must know how to
program the function and compile it. Without using \emph{CUDA} we could simply
use function pointers so that any arbitrary function can be passed and no need
for recompilation is needed. However, function pointers are extremely slow in \emph{CUDA}
since the function cannot be \emph{inlined} and we introduce massive overhead by
adding function calls to the stack.

One option is to implement a program to parse ODE systems with a strict syntax,
produces the \emph{CUDA} kernel compiles it and dynamically links it to the
program. It should be possible to do from \emph{Julia}.
