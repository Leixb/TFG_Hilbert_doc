%! TEX root = **/010-main.tex
% vim: spell spelllang=en:

\section{Conclusions}%
\label{sec:conclusions}

With this project we were able to develop and implement a fast method to assist
in the detection of limit cycles by leveraging the computational power of modern
GPUs.  The program is much faster than finding limit cycles by integrating a
trajectory until it converges, which can take a lot of time depending on the
initial conditions and the system.

Initially the program was implemented in the \emph{Julia} programming language,
but the \emph{CUDA} part of the program was finally implemented in \emph{C} do
to the limited documentation and capabilities of \emph{Julia}'s \emph{CUDA} API.
Despite being written in \emph{C}, its was integrated into \emph{Julia} through
an \emph{ABI} which allows the user to use the advanced abstraction and libraries
of \emph{Julia} to visualize and further process the data obtained.

The program can also be used with a powerful GPU to perform a parameter search
to find interesting systems. The parameter search performed has been tiny
compared to the initial objective due to the limited resources and time
available but can be easily scaled to use multiple GPUs as shown in
\cref{sub:multi-gpu}. The nature of the algorithm and its simplicity allows
to scale it indefinitely.

Unfortunately, not all limit cycles on a system are found
reliably, there are a couple improvements that could be made in the future to
fix this issue. In \cref{sec:future} there is a discussion on the future of the
project.

All in all, the project is an initial proof of concept on how the processing power
of a GPU can be used in the study of limit cycles. This opens the door for further research
of the topic using more advanced methods.
