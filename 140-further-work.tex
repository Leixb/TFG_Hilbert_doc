%! TEX root = **/010-main.tex
% vim: spell spelllang=en:
%
\section{Further work}%
\label{sec:future}

Despite obtaining promising results, many improvements can be made. Specially
regarding possible analytical methods that can help narrowing down locations of
possible limit cycles. Another area of the project that could be improved is the
integrator methods for stiff equations.
There are hundreds of well researched integration methods that far outperform the
simple \emph{Runge Kutta} methods used in the project.
Porting the implicit integrators from GNU Scientific Library to \emph{CUDA}
as some researchers have done with other parts of the library \cite{rodrigo_gnu_2019}
could enable much better integration for stiff systems.


The \emph{UAB} has an open source program named \emph{P4}
\cite{saleta_oscarsaletap4_2018,saleta_computer_2018} to analyze ODE systems
numerically and analytically. It does find limit cycles by computing
trajectories between two points given by the user until it reaches a cycle. This
method is quite slow, in fact in~\cite{dumortier_examples_2006} there is a
explicit mention on how you should specify points very close by and reduce the
precision of the computation since otherwise the limit cycle detection may take
a lot of time. The following quote from page 267 illustrates the issue:

\begin{quote}
    A \emph{Searching for limit cycles} window  appears  with  a  time  bar
    which  should show the time left for computing but whose most useful
    application is to stop searching, since it may easily delay a lot before or
    after finding a limit cycle.
\end{quote}

Maybe our approach could be added as an alternative to assist in finding limit cycles.
