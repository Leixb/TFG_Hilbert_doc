%! TEX root = **/010-main.tex
% vim: spell spelllang=en:

\section{Final sustainability report}
\label{sec:sustainability-final}

\subsection{Environmental Impact}

\paragraph{PPP (project put into production)}

The environmental impact can be quantified in the number of hours of usage of
the server used to run the program.  According to the server logs, there have
been around 170 hours of server usage from the account I used. It is difficult
to estimate accurately how this translates to energy usage, but most of the time
was spent developing the program and running tests on a single system. The two
parameter searches performed where also quite small and ran for around 5 and 12
hours each using around 200 Watts on the GPU. All these estimates do not take
into account the computing time spent on my own machine or the multi-GPU
execution on the \emph{boada} server, but overall the environmental impact was
quite small.

If I were to carry the project out again, I would probably be able to use less computing
resources in the initial stages of the development of the program. During the first
iterations of the development cycle, I ran various tests that didn't really work and
took quite some time to compute.

\paragraph{Exploitation}

The main resources needed to use the project are the GPUs needed. As of now, there are
two possible use cases for the program: helping find limit cycles of a unique
system, or performing a parameter search
to find systems with interesting configurations of limit cycles.
The former can done with a simple \emph{CUDA} capable GPU and requires \emph{minimal}
resources. The latter is much more costly depending on the number of parameters to search,
it could require a powerful GPU or even a bunch of GPUs.

The program makes good use of the GPU resources, as such, it should be more energy
efficient than running the same method on a CPU or the method of computing a single trajectory
from a point until convergence used to find limit cycles with a CPU.

\paragraph{Risks}

The program is unable to find limit cycles in some cases, so it may have to be improved
in the future. If the proposed
new integrators to be able to find those cycles are implemented, there could
be a change in the resources needed to run the program. Nonetheless, it will probably
have minimal impact, since now there is still a need to run inefficient CPU code
to search stiff areas of a system.

\subsection{Economic Impact}

\paragraph{PPP}

Overall, the initial budget outlined in \cref{sec:budget} was a good estimation of the
cost of the project, there were no major problems that required modifications to the
budget nor was there any need for the contingency budget.

\paragraph{Exploitation}

As we pointed out in the \emph{PPP} section, the main cost of the project is the
computational resources to run it (what GPUs and how many are needed) as well as their
power consumption.

Right now the project needs some polishing to reach its full potential, therefore an
update may be needed in the future, probably to address the improvements
discussed in \cref{sec:future}. This will require human resources to work on the program
and maintain it.

\paragraph{Risks}

Currently, the program has been tested with a very few systems. A rigorous test on the
numerical stability of the integrators used was performed, but that is not enough to
guarantee there may not be issues with some systems.


\subsection{Social Impact}

\paragraph{PPP}

Undertaking this project I have learned a lot about general purpose graphics
processing unit programming (GPGPU) and scientific computing.

\paragraph{Exploitation}

As we mentioned on \cref{sec:future} there are researchers using similar tools
to explore constructs on ODEs and quick identification of limit cycles could
help speedup their research.

The program developed in the project does identify limit cycles quickly, but has trouble
when dealing with some trajectories. It can assist on finding limit cycles, but it is not
a complete replacement to other methods. Some parameter searches were performed but all of
them only managed to find 3 limit cycles.

\paragraph{Risks}

This project uses the proprietary \emph{CUDA} platform. Therefore, it can only be executed
on \emph{Nvidia} GPU. This makes the users dependent on this vendor since they cannot
run the program on other GPUs.
